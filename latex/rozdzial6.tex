\chapter{Proponowany model zjawiska}

\section{Cele modelu}
Celem modelu jest zasymulowanie rozwoju populacji zwierząt. Stworzony przez nas program pozwala na zasymulowanie populacji modelu drapieżnik-ofiara w postaci jednego gatunku ofiary i  jednego gatunku drapieżnika oraz jednego gatunku ofiary dwóch gatunków drapieżników. Zwierzęta mogą poruszać się po zadanym obszarze nie opuszczając go. 

Gdy osoba rozpocznie symulacje może na bieżąco wyświetlać informacje na temat liczby iteracji, ilości populacji każdego z gatunków. Dodatkowo przedstawiony jest widok mapy migracji zwierząt.

\section{Działanie algorytmu} 
W pierwszym etapie działania programu generowanie są populacje zadanych gatunków. Stworzone populacje są rozmieszczane losowo na mapie w ilości uzależnionej od zadanych przez nas współczynników. Następnie na każdym z pól tworzona jest ilość pożywienia dla ofiar, czyli rośliny. 

Po inicjalizacji modelu rozpoczyna się główna część symulacji. W każdej iteracji każde zwierzę obejmuje jakąś strategię działania.
\begin{enumerate}
\item Wspólne na wszystkich zwierząt
	\begin{itemize}
		\item Kopulacja
		\item Wydawanie na świat potomstwa
	\end{itemize}
\item Dla ofiar
	\begin{itemize}
		\item Poszukiwanie i spożywanie roślin
		\item Ucieczka przed drapieżnikami
		\item Odpoczynek
	\end{itemize}
\item Dla drapieżników
	\begin{itemize}
		\item Polowanie na ofiary
		\item Ucieczka przed naddrapieżnikami
	\end{itemize}
\item Dla naddrapieżników(michała)
	\begin{itemize}
		\item Polowanie na ofiary i drapieżników
	\end{itemize}
\end{enumerate}

Poszczególne zachowania uzależnione są od aktualnego stanu danego zwierzęcia oraz jego otoczenia.

\begin{itemize}
\item Wspólne strategie zwierząt
	\begin{itemize}
		\item Jeżeli samica jest w kluczowym stadium ciąży musi wydać na świat potomstwo.
		\item Jeżeli zwierzę jest zagrożone stara się uciec.
		\item Jeżeli współczynnik głodu jest krytyczny ofiara usiłuje zdobyć pokarm.
		\item Jeżeli współczynnik pożądania seksualnego jest krytyczny i osobnik jest wystarczająco dojrzały rozmnaża się z drugim przedstawicielem swojego gatunku przeciwnej płci na danym polu lub szuka partnera w swojej okolicy.
		\item Jeżeli zwierze ma wszystkie współczynniki w normie, odpoczywa.
	\end{itemize}
	
\item Szczególne dla ofiary
	\begin{itemize}
		\item Jeżeli na polu jest drapieżnik, który poluje na ofiarę stara się uciec z danego pola
		\item Jeżeli współczynnik głodu jest krytyczny ofiara żywi się pokarmem na aktualnym polu lub jeśli pole jest pozbawione roślin ofiara szuka ich w swoim najbliższym otoczeniu.
	\end{itemize}
	
\item Dla drapieżnika
	\begin{itemize}
		\item Jeżeli na jego polu jest drapieżnik drugiego stopnia, drapieżnik stara się przemieścić w dogodniejsze dla siebie miejsce
		\item Jeżeli współczynnik głodu jest krytyczny drapieżnik poluje na ofiarę znajdującą się na jego polu lub szuka jej w swoim najbliższym otoczeniu.
	\end{itemize}
	
\item Dla drapieżnika drugiego poziomu
	\begin{itemize}
		\item Jeżeli współczynnik głodu jest krytyczny drapieżnik drugiego stopnia w pierwszej kolejności stara się upolować drapieżnika, który jest preferowanym gatunkiem. W razie jego braku poluje na przedstawiciela gatunku reprezentowanego przez ofiarę.
	\end{itemize}
\end{itemize}

!!!!!
predator moze rozmnazac sie tylko w warunkach sprzyjajacych
czyli w polach na ktorych moze znalezc pozywienie
w innym wypadku jest eksplozja
i zaganianie preyow do rogow

\section{Symulacja zjawiska}

\subsection{Narzędzi}
Na potrzeby symulacji została stworzona aplikacja okienkowa, która wizualizuje efekty działania aplikacji.
Do zaimplementowania programu posłużyliśmy się językiem Java, a do wyświetlenia danych użyliśmy biblioteki graficznej JavaFX.

\subsection{Wyniki}

Wyniki kilku symulacji zapisaliśmy do pliku tekstowego, a następnie stworzyliśmy na tej podstawie odpowiednie wykresy. W celu późniejszej możliwości walidacji przyjętego przez nas modelu.

\subsubsection{Wyniki dla 2 gatunków- ofiara, drapieżnik}


\subsubsection{Wyniki dla 3 gatunków- ofiara, drapieżnik i drapieżnik drugiego stopnia}




\section{Wnioski}

Podstawowym wnioskiem jaki zauważyliśmy podczas tworzenia projektu jest ogromny wpływ zwięszkania się współcznyika głodu dla poszc9zególnych 

ruchy zwiększenie głodu po iteracji sznasa zabicia


